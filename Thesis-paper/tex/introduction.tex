\chapter*{Introduction}

Wearing a mask in public has become one of the most important steps towards fighting the ongoing pandemic.
Governing bodies have made numerous steps towards ensuring that people uphold this new custom.
Although governments have created a number of laws which prohibit being mask less in public, some people intentionally forego, or simply forget to uphold these laws.
Being able to algorithmically recognize someone who does not abide these rules would decrease the spread of the pandemic.

Image recognition is a widely researched topic in modern computer science. 
There has been a number of different approaches and methods that are capable of recognizing objects of varying complexities.

Machine learning has proved to be one of the most powerful model among image recognition methods. The World Health Organisation (WHO) has identified two things that help curve the spread, namely; wearing or masks and social distancing. It is not as easy to enforce such manually, hence the need to automate such is paramount. Research such as \cite{chowdary2020face} has also been a key in helping come up with possible solutions, their system has managed to get an accuracy of over 99\% in training. The main goal of this work is to thus develop a working model to help detect faces that are wearing masks and those that are not wearing masks. this model is to use a transfer learning model called Inception created by Google Brain. 

The paper is divided into six sections with each section heaving subsections. The first sections will give an overview of some of the related work that has been done, second section will give us the method used including the datasets and the network used. Further sections will underline the tools and libraries used, results and the discussion. The last section will then give us the conclusion.


\section*{Motivation}
The motivation behind the idea and development of this system basically came from the struggles I, personally, was having with the first lockdown. When Covid-19 first hit, I was far from family, no one really knew what was happening nor how to deal with it. University was closed, People sent home and we were the only ones left at the Dormitories and everything suddenly changed. To keep myself busy and hopefully contribute to the science world that was trying to come up with systems that could be used to curve the spread of the virus. wearing of masks was suggested as a potential act that could be useful and I happened to stumble upon a dataset and a task on Kaggle.com by Prajna Bhandary. She compiled this big chunk of dataset of faces with masks as well as faces without masks and the task and challenge seemed like a good way to keep my head up and do something useful. Other motivation came from my zeal to sharpen my Python skills, it seemed like a good way to put my test into practice as well as using deep models that I had previously studied and worked with during my internship. It was honestly a mental escape for me working on this system and I am glad I started working on it when I did.

\section*{Goal}
Over the years, there has been an escalating advancement of deep learning and face recognition in various sectors of the society. Ours here is focused on the masks that we all have to wear in public places

This system seeks to reach the following goals 
\begin{itemize}
    \item Identify faces that are wearing masks
    \item Identify faces that are not wearing masks
    \item Be easy to use
    \item Help curve the spread of the virus
\end{itemize}

\section*{Problem statement}

The idea and reason for a technical system is because it can be hard to physically monitor pupils in public spaces weather they are wearing masks or not. My area of concern were To understand the current issues being faced by the society, how an IT system can be used to help communities and for the system to be useful even in the future.